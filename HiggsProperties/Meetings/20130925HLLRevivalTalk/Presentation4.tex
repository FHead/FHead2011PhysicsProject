\documentclass[11pt,t]{beamer}
\usepackage[latin1]{inputenc}
\usepackage{amsmath}
\usepackage{amsfonts}
\usepackage{amssymb}
\usepackage{graphicx}
\usepackage{tikz}
\usetheme{default}
\setbeamertemplate{footline}[frame number]

\title{Measuring Higgs anomalous HVV couplings with maximum likelihood fit}
\subtitle{Status and plans - draft(v1) for Wednesday}
\author{Yi Chen}
\institute{California Institute of Technology\\~\\Caltech group meeting}
\date{\today}

\begin{document}

\begin{frame}[plain]
   \titlepage
\end{frame}

\begin{frame}{Table of contents}
   \begin{itemize}
   \item Status
   \item Quick overview of strategy and convolution
   \item Some words on bug fixes and improvements
   \item Near-term plans
   \item Summary
   \end{itemize}
\end{frame}

\begin{frame}{Status}
   \begin{itemize}
   \item Many rounds of bug fixing in the past couple of months - all known bugs/features fixed.
   More details in later slides.
   \item Rethink many steps to make sure our result is robust
      \begin{itemize}
      \item For example the normalization of one set of pdf can be achieved to O(0.1\%) in 10 minutes
      as opposed to hours or even days
      \end{itemize}
   \item Once we are happy with the analysis, carry out the measurement on real data
   will take half a day from beginning to end, reproducing all numbers from scratch.
   \end{itemize}
\end{frame}

\begin{frame}{Quick overview of strategy}
   \begin{itemize}
   \item Construct likelihood as a function of Lagrangian parameters
      \begin{itemize}
      \item Obtain differential cross sections at generator level
      \item Perform an integral event by event convoluting generator level distribution
      with parameterization of detector response for leptons
      \item Perform a calculation of the pdf normalization
      \end{itemize}
   \item Perform maximization on data to extract parameters of interest
   \item Current Lagrangians that we have differential cross sections calculated:
   \begin{eqnarray}
   L \simeq M_Z^{-2} A_1^{ZZ} X Z_\mu Z^\mu + A_2^{ZZ} X Z_{\mu\nu} Z^{\mu\nu} + A_3^{ZZ} X Z_{\mu\nu} \tilde{Z}^{\mu\nu} + ...\nonumber\\
   + A_2^{ZA} X Z_{\mu\nu} F^{\mu\nu} + A_3^{ZA} X Z_{\mu\nu} \tilde{F}^{\mu\nu}\nonumber + ...\nonumber\\
   + A_2^{AA} X F_{\mu\nu} F^{\mu\nu} + A_3^{AA} X F_{\mu\nu} \tilde{F}^{\mu\nu}\nonumber + ...\nonumber
   \end{eqnarray}
   \item $Z \rightarrow 4l$ coming soon.  Will be important cross check to measure Z spin.
   \end{itemize}
\end{frame}

\begin{frame}{Quick overview of convolution}
   \begin{itemize}
   \item Start from a consistent equation:
   \begin{equation}
      P^R(\vec{X} | \vec{A}) = \int P^G(\vec{X}' | \vec{A}) T(\vec{X} | \vec{X}') d \vec{X}'\nonumber
   \end{equation}
   \item Change of variables to our convenience:
   \begin{eqnarray}
      P^R(\vec{X} | \vec{A}) &=& \int P^G(\vec{X}' | \vec{A}) T(\vec{P}^G_\parallel | \vec{X}') \dfrac{|J_M^G|}{|J_M^R|} d \vec{P}^G_\perp d \vec{P}^G_\parallel\nonumber\\
         &=& \int P^G(\vec{X}' | \vec{A}) T(\vec{c}^G | \vec{X}') \dfrac{|J_M^G J_c^G|}{|J_M^R J_c^R|} d \vec{c}^G\nonumber\\
         &=& \int P^G(\vec{X}' | \vec{A}) T(\vec{c}^G | \vec{X}') \dfrac{|J_M^G J_c^G|}{|J_M^R J_c^R|} |J_B| d M_1^2 d M_2^2 d c_1 d c_3.\nonumber\\
        (signal)  &=& \int P^G(\vec{X}' | \vec{A}) T(\vec{c}^G | \vec{X}') \dfrac{|J_M^G J_c^G|}{|J_M^R J_c^R|} |J_S| d M_1^2 d M_2^2 d c_1 d M_H^2.\nonumber
   \end{eqnarray}
   \end{itemize}
\end{frame}

\begin{frame}{Quick overview of convolution}
   \begin{itemize}
   \item $|J_M^G|$ and $|J_M^R|$ changes from center-of-mass variables with production variables to lepton vector basis.
   Each of them are 12x12 Jacobians.
   \item $|J_c^G|$ and $|J_c^R|$ takes us from lepton vector basis to smearing basis, which is what the detector responses are defined in
   \item $|J_B|$ changes from the smearings to di-lepton masses, which is for convenience (better book-keeping).
   \item In terms of signal things are slightly different, we change to $d M_1^2 d M_2^2 d M_H^2 d c_1$ to take care of the
   narrow width approximation (ie., factor of $\delta(M_H^2 - 126^2)$ in the differential cross section).
   \item Once the narrow width approximation is imposed, in $c_1 c_3$ (smearing of two of the leptons)
   plane allowed phase space reduces to
   1 dimension (which happens to form an ellipse).
   \item All these factors were wrong or missing couple months ago...
   \end{itemize}
\end{frame}

\begin{frame}{Some discussion on strategy}
   \begin{itemize}
   \item There are near-cancellation of terms which won't show up easily in many places except in final fit biases.
   Projections will look fine, but bias on the parameters.  This is why things look reasonable before.
   A stronger test with likelihood distribution is successful in catching these bugs.
   \item Dimensionality isn't the point as long as correct likelihood is used in a correct way.
   For example generator level likelihood is the same as calculated differential cross sections,
   which is why in generator level, test from different framework checks out.
   The analogy would be using reco-level likelihood for different analysis frameworks, which
   is not the same as using combination of gen-level differential cross section as test statistics in
   reco-level analysis. \textbf{Need to find out a way to say this so as not to annoy some people}
   \end{itemize}
\end{frame}

\begin{frame}{Some discussion on strategy}
   \begin{itemize}
   \item Production variables enter through detector acceptance effects.  To reduce dependence on
   production variables, we average out the 4 dimensions irrelevant to $HVV$ vertex ($12 \rightarrow 8$)
   \item Systematics can be easily incorporated with nuisance parameters
   \item This approach is the simplest conceptually
   \item We fit for ratios/fractions and not worry about absolute rates
   \item Can fit simultaneously for all possible parameters as long as we have enough events
   \item Deal with fake background by making a big template + large systematics to cover possible variations.
   \item \textbf{Whatever else we want to talk about here...}
   \end{itemize}
\end{frame}

\begin{frame}{Validation of convolution}
   \textbf{Let's add some projection plots here}
\end{frame}

\begin{frame}{Validation of convolution}
   \begin{itemize}
   \item Distribution of likelihoods must match between events generated by the pdf (RECO level)
   and Madgraph samples after detector effects
   \item This is a stronger check than individual projections
   \end{itemize}
   \begin{figure}
   \includegraphics[width=0.5\textwidth]{BackgroundLikelihoodAfterEFix.pdf}
   \includegraphics[width=0.5\textwidth]{SignalLikelihoodAfterEFix.pdf}
   \caption{Likelihood comparison between generated samples and Madgraph samples at RECO level.
   Left: Background, Right: SM signal}
   \end{figure}
\end{frame}

\begin{frame}{Non-exhaustive list of bug fixes and improvements}
   \begin{enumerate}
   \item Fixing of the signal ellipse
   \item Inclusion of ``tip'' region
   \item Rework 12x12 Jacobians, with validations
   \item Fixing Jacobian pairs
   \item Jacobian factor from mass scan directions
   \item Fit parameterization 
   \item Removed unnecessary approximations
   \item Rethink normalization with huge speedup
   \item Mass attractors
   \item Reboot mechanism for added robustness
   \item Scan directions to decorrelate grid directions
   \item Adaptive mass scans
   \item[14.] ...
   \end{enumerate}
\end{frame}

\begin{frame}{Timeline and plans}
   Statistics on toy events are accumulating right now...  Plan to finish these basic analysis parts as soon as possible.
   Result with lower stats look reasonable.
   \begin{itemize}
   \item Bias studies - whether we can extract things correctly (with pull distributions to show that errors are reasonable)
   \item Projection to different amount of statistics
   \item etc.
   \end{itemize}
   Present details bit by bit for scrutiny in this group from now on, hopefully we can open the box in a few months.

   \textbf{Show some 100-dataset pull plots here}
\end{frame}

\begin{frame}{Summary}
   \begin{itemize}
   \item Rounds of bug fixes done.  Finally ready to move on again.
   \item Various improvements have been implemented that speed things up and make things more robust
   \item Toy results will appear soon; and in the meantime the analysis will be presented in pieces for scrutiny.
   \item We are preparing a document on all the details of calculation.
   \end{itemize}
\end{frame}

\begin{frame}{Backup slides ahead}
\end{frame}

\begin{frame}{Blank page}
\end{frame}


\end{document}




