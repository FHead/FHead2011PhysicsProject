\documentclass[12pt,a4paper]{article}
\usepackage[latin1]{inputenc}
\usepackage{amsmath}
\usepackage{amsfonts}
\usepackage{amssymb}
\usepackage{makeidx}
\author{Yi Chen}

\begin{document}

\section{One dimensional simultaneous fit}

In the 2010 version of the razor analysis, a set of simple exponential fits to the tail of razor variables is performed independently.
The parameters $a$ and $b$, specific to each process, are extracted by a linear fit to the exponents.
One minor drawback for this approach is that the set of events are not mutually exclusive between different fits, and the correlation between parameters is not properly taken into account.
To get a better handle on this correlation, one-dimensional simultaneous fit is developed.
Instead of doing fits in each $R > R0$ region, the dataset is divided into different exclusive $R$ bins, where each bin is modelled by difference of two exponential functions.
On top of this separation, we can constrain the exponent to be linear as a function of square of R cut, thereby extracting the $a$ and $b$ parameters directly.


\end{document}